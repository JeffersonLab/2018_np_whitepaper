%Flog usefulness of SciDAC software etc.

Nuclear Physics is a diverse field with linkages to many areas of
research and experimentation, including the structure of the hadrons,
and the properties of the nuclei composed of protons and neutrons.
There are existing and new generations of experiments within the US,
and also worldwide, dedicated to explaining these properties as laid
out in the 2015 NSAC Long Range Plan for Nuclear Physics.  The recent
12GeV upgrade of Jefferson Lab (JLab) and the planned Electron Ion
Collider (EIC) will peer into the internal structure of hadrons and
look for the possible existence of exotic states of matter. The
Facility for Rare Isotopes program (FRIB) will clarify how subatomic
matter organizes itself and how nuclei emerge.

Over the years, this impressive level of experimentation has provided
numerous discoveries that have led to the development of the
fundamental theory that describes the strong interactions -- Quantum
Chromodynamics (QCD). This theory, when combined with the electroweak
interactions, underlies all of nuclear physics, from the spectrum and
structure of hadrons to the most complex nuclear reactions. However,
many aspects of nuclear physics are dictated by the regime of QCD in
which its defining feature--asymptotic freedom--is concealed by
confinement and by the complicated structure of the quantum
vacuum. The numerical technique of {\it Lattice QCD} is the only known
way to perform {\it ab initio} QCD calculations of strong interaction
quantities in this regime. The ability to compute the properties of
matter, with quantifiable uncertainties, is necessary to establishing
a bridge between theory and experiments, and vital to progress in the
field.  

Lattice QCD (LQCD) is a technique in which space and time are
discretized and strong interaction quantities are calculated by
large-scale numerical Monte-Carlo integration, and in which
approximation effects can be systematically removed. LQCD calculations
have been at the forefront of leadership computing resources for
decades, and the ambitious plans put forth in this white paper will
require much larger computing capabilities. To this end, the SciDAC
programs have been essential to achieving high performance on new hardware
architectures, and LQCD calculations have lead the development and
adoption of new computing paradigms, including the use of graphical
processing units. Local computing resources under the USQCD Initiative
have also been essential to leveraging the advances using the
leadership facilities. Looking to the future, Exascale computing resources
will be required, and the software development efforts under the
Exascale Computing Project are paving the way for new calculations
heretofore not obtainable.


This white paper provides a roadmap for on-going and future science
programs that will have a profound impact on our understanding of Nature and
that of hadrons and nuclei.

