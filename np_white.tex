\begin{filecontents}{leer.eps}
%!PS-Adobe-2.0 EPSF-2.0
%%CreationDate: Mon Jul 13 16:51:17 1992
%%DocumentFonts: (atend)
%%Pages: 0 1
%%BoundingBox: 72 31 601 342
%%EndComments

gsave
72 31 moveto
72 342 lineto
601 342 lineto
601 31 lineto
72 31 lineto
showpage
grestore
%%Trailer
%%DocumentFonts: Helvetica
\end{filecontents}


%\documentclass[aps,prd,nofootinbib,floatfix,superscriptaddress,preprint,tightenlines]{revtex4-1}
\documentclass[epj,nopacs]{svjour}

\usepackage{amsfonts}
\usepackage{amsmath}
\usepackage{amssymb}
\usepackage{bm}
\usepackage{cite}
\usepackage{graphicx}
\usepackage[pdftex]{color}
%\usepackage[sort&compress]{natbib}
\usepackage[colorlinks=true,linkcolor=blue,filecolor=blue,urlcolor=blue,citecolor=blue,pdftex,plainpages=false]{hyperref}
\usepackage{sidecap}
\usepackage{wrapfig}

\usepackage[utf8]{inputenc}

\begin{document}

\title{Hadrons and Nuclei}

\author{William Detmold\inst{1}\thanks{Editor, \email{wdetmold@mit.edu}} 
 \and Robert G. Edwards\inst{2}\thanks{Editor, \email{edwards@jlab.org}}
 \and Jozef J. Dudek\inst{2,3}
 \and Michael Engelhardt\inst{4}
 \and Huey-Wen Lin\inst{5}
 \and Stefan Meinel\inst{6,7}
 \and Kostas Orginos\inst{2,3}
 \and Phiala Shanahan\inst{1}\\
\\
(USQCD Collaboration)}

\institute{Massachusetts Institute of Technology
 \and Jefferson Lab
 \and William \& Mary
 \and New Mexico State University
 \and Michigan State University
 \and University of Arizona
 \and RIKEN BNL Research Center}

\date{Received: date / Revised version: date}

\abstract{
  In 2018, the USQCD collaboration’s Executive Committee organized several subcommittees to recognize future opportunities and formulate possible goals for lattice field theory calculations in several physics areas.  The conclusions of these studies, along with community input, are presented in seven whitepapers~\cite{Bazavov:2019lgz,Brower:2019oor,Cirigliano:2019jig,Detmold:2019ghl,Joo:2019byq,Kronfeld:2019nfb,Lehner:2019wvv}.
Here, we discuss opportunities for lattice QCD calculations related to the structure and spectroscopy of hadrons and nuclei. An overview of recent lattice calculations of the structure of the proton and other hadrons is presented along with prospects for future extensions. Progress and prospects of hadronic spectroscopy and the study of resonances in the light, strange and heavy quark sectors is summarized. Finally recent advances in the study of light nuclei from lattice QCD are addressed and the scope of future investigations that are currently envisioned is outlined.
}

\maketitle

%\newpage
%\tableofcontents
%\newpage 

\section*{Executive summary}
\input executive

\section{Introduction}
\label{sec:intro}
\input intro

\section{Hadron Structure}
\label{sec:hadronstructure}
\input hadstruct

\section{Hadron Spectroscopy}
\label{sec:hadronspectroscopy}
\input hadspec

\section{Nuclear Spectroscopy, Interactions and Structure}
\label{sec:nuclear}
\input nuclear

\section{Future Opportunities}
\label{sec:future}
\input future


%%%%%%%%%%%%%%%%%%%%%%%%%%%%%%%%%%%%%%%%%%%%%%%%%%%%%%%%%%%%%%%%%%%%%%%%%%%%%%%%%
\begin{acknowledgement}
JJD and RGE acknowledge support from the U.S. Department of Energy contract DE-AC05-06OR23177, under which Jefferson Science Associates, LLC, manages and operates Jefferson Lab. 
\end{acknowledgement}


\bibliographystyle{epj}
\bibliography{usqcd-wp,hadspec,hadstruct,nuclear,future}

\end{document}
