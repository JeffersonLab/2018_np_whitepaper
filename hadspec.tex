
\begin{itemize}
	\item Context of hadron spectroscopy in terms of renaissance of experimental investigations and discoveries (XYZ states, tetra- and penta-quark objects)
	\item light hadron spectroscopy relevant for GlueX and other experiments. 
	\item dealing with unstable resonances and coupled channels
	\item extension of formalism to resonances with  three- and higher-body final states
	\item future emphasis on baryon resonances
	\item transition form facotrs for understanding decay patterns
	\item calculations of matrix elements in resonances to understand their structure
	\item try to develop an understanding of XYZ states
	\item make further predictions for baryons containing charm and bottom quarks as LHCb continues to probe this sector
\end{itemize}





\input dudek/dudek





\subsection{XYZ states}

\hide{
	Within the charm sector, the lattice QCD methods described above can be brought to bear on the question of flavor exotics and the other excess `XYZ' states. There have been suggestions that at least some of the observed enhancements arise due to the kinematics of the three-body production process (e.g. $e^+e^- \to J/\psi \, \pi \pi$ or $B \to K \, \psi' \pi$), rather than being due to a true two-body resonance. The lattice calculation has the advantage that it does not need to consider a particular production process, rather it can determine the two-body scattering amplitude directly, and thus investigate the resonance content.
{\color{red} some refs to partial work done ?}

{\color{red} Also discuss $bb\bar{u}\bar{d}$ tetraquarks here \cite{Francis:2016hui}?}

}
\subsection{Charmed and bottom baryon spectroscopy}

Baryons containing heavy quarks are very interesting systems in QCD, because approximate heavy-quark flavor and spin symmetries
constrain their spectrum and dynamics \cite{Korner:1994nh, Manohar:2000dt}.
% Singly charmed and singly bottom baryons exhibit a similar spectrum of excitations of the light degrees of freedom. Interactions with the
% spin of the heavy quark, and hence the hyperfine splittings, are suppressed by $1/m_Q$.
A particularly interesting symmetry emerges for doubly heavy baryons: in the large-mass limit, the two heavy quarks
are expected to form a point-like diquark that acts like a single heavy antiquark, and the light degrees of freedom behave as in a
heavy-light meson \cite{Savage:1990di, Brambilla:2005yk}.

With the current operation of the Large Hadron Collider, charm and bottom baryons are being produced in unprecedented quantities.
This has led to several discoveries in the last few years \cite{Chatrchyan:2012ni, Aaij:2012da, Aaij:2014yka, Aaij:2016jnn, Aaij:2017ueg,  Aaij:2017vbw, Aaij:2017nav}, with many more expected in the future. Lattice QCD
can predict the masses, can help assign $J^P$ quantum numbers, and can also provide information on the structure and decay rates.

Until recently, the only experimental observation of a doubly-heavy baryon candidate was that by the SELEX Collaboration in 2002,
interpreted as the $\Xi_{cc}^+$ \cite{Mattson:2002vu}. Lattice QCD calculations consistently gave a $\Xi_{cc}$ mass approximately
100 MeV higher than the SELEX result (see Refs.~\cite{Brown:2014ena, Padmanath:2015jea, Bali:2015lka, Alexandrou:2017xwd} for calculations published since 2014).
In 2017, the LHCb Collaboration reported the observation of the $\Xi_{cc}^{++}$ with more than 12$\sigma$ significance \cite{Aaij:2017ueg} at a mass consistent with the lattice QCD predictions.
Note that the isospin splitting between the $\Xi_{cc}^{++}$ and $\Xi_{cc}^{+}$ has also been calculated using lattice QCD+QED to be $2.16(11)(17)$ MeV \cite{Borsanyi:2014jba}, confirming that
the structure seen by SELEX cannot be the $\Xi_{cc}^{+}$.

Another recent result from LHCb is the observation of five narrow $\Omega_c$ resonances decaying to $\Xi_c^+ K^-$ \cite{Aaij:2017nav}. The $J^P$ quantum
numbers of these resonances were not measured, but were assigned by comparing to lattice QCD in Ref.~\cite{Padmanath:2017lng}.

The available lattice QCD predictions of ground-state heavy-baryon masses still have large uncertainties of order 20 MeV or more \cite{Brown:2014ena, Padmanath:2015jea, Bali:2015lka, Alexandrou:2017xwd},
and there is a lot of potential for improvements in the future. Predictions for isospin splittings in lattice QCD+QED are also needed. In addition, the decay widths of selected excited states (those coupling to two-body channels only)
can be calculated using the L\"uscher method. Finally, a significant challenge is posed to lattice QCD by the observation of $J/\psi\: p$ pentaquark resonances by LHCb \cite{Aaij:2015tga}.


% Recent experimental discoveries include the $\Xi_b^{\prime}$ and $\Xi_b^{*0}$ \cite{Chatrchyan:2012ni, Aaij:2014yka, Aaij:2016jnn},
% the $\Xi_{cc}^++$ \cite{Aaij:2017ueg}, and several excited $\Lambda_c$, $\Omega_c$, and $\Lambda_b$ states \cite{Aaij:2012da, Aaij:2017vbw, Aaij:2017nav}.

% 
% 
% In the absence of experimental data, lattice QCD calculations of triply-heavy baryon spectra \cite{Meinel:2012qz, Padmanath:2013zfa} have also provided guidance for
% other theoretical studies of these systems \cite{Vijande:2015faa, Qin:2018dqp, Serafin:2018aih}
% 



