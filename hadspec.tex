
\begin{itemize}
	\item Context of hadron spectroscopy in terms of renaissance of experimental investigations and discoveries (XYZ states, tetra- and penta-quark objects)
	\item light hadron spectroscopy relevant for GlueX and other experiments. 
	\item dealing with unstable resonances and coupled channels
	\item extension of formalism to resonances with  three- and higher-body final states
	\item future emphasis on baryon resonances
	\item transition form facotrs for understanding decay patterns
	\item calculations of matrix elements in resonances to understand their structure
	\item try to develop an understanding of XYZ states
	\item make further predictions for baryons containing charm and bottom quarks as LHCb continues to probe this sector
\end{itemize}





\input dudek/dudek





\subsection{XYZ states}

\hide{
	Within the charm sector, the lattice QCD methods described above can be brought to bear on the question of flavor exotics and the other excess `XYZ' states. There have been suggestions that at least some of the observed enhancements arise due to the kinematics of the three-body production process (e.g. $e^+e^- \to J/\psi \, \pi \pi$ or $B \to K \, \psi' \pi$), rather than being due to a true two-body resonance. The lattice calculation has the advantage that it does not need to consider a particular production process, rather it can determine the two-body scattering amplitude directly, and thus investigate the resonance content.
{\color{red} some refs to partial work done ?}
}
\subsection{Charmed and bottom baryons spectroscopy} 

\hide{Progress on experimental front at LHCb}






