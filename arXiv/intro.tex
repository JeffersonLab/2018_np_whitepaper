Hadrons and nuclei make up the bulk of everyday matter and are the objects that are detected in experiments at accelerators and colliders. Yet hadrons and nuclei are complex composite objects that emerge from the underlying strong interactions between quarks and gluons, the fundamental degrees of freedom of Quantum Chromodynamcis (QCD). Describing this compositeness is challenging as the couplings between quarks and gluons become large at the relevant energy scales, and the perturbative approach that works well for QED and for QCD at high energy breaks down.
%
In this USQCD collaboration whitepaper, we discuss the application of the numerical techniques of lattice QCD (LQCD) to calculations of the non-perturbative properties of hadrons and nuclei. We summarize the  recent accomplishments of LQCD (and USQCD in particular) in this domain and discuss future goals and opportunities in the context of current and future experiments. There  are numerous synergies between the topics discussed here and those discussed in the six companion USQCD whitepapers
\cite{Bazavov:2018qcd,Brower:2018qcd,Davoudi:2018qcd,Joo:2018qcd,Kronfeld:2018qcd,Lehner:2018qcd}.
which are highlighted in the following.

