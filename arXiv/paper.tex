\documentclass[hyperpdf,aps,prd,superscriptaddress,nofootinbib,preprint,preprintnumbers,floatfix,tightenlines]{revtex4-1}

\pdfoutput=1   

\usepackage{amsfonts}
\usepackage{amsmath}
\usepackage{amssymb}
\usepackage{bm}
\usepackage{graphicx}
\usepackage[pdftex]{color}
\usepackage[sort&compress]{natbib}
\usepackage[colorlinks=true,linkcolor=blue,filecolor=blue,urlcolor=blue,citecolor=blue,pdftex,plainpages=false]{hyperref}
\usepackage{sidecap}
\usepackage{wrapfig}

\usepackage[utf8]{inputenc}

\newcommand{\hide}[1]{{#1}}

\begin{document}

\title{Hadrons and Nuclei}
%
\author{William Detmold (editor)}
\affiliation{Massachusetts Institute of Technology}
%
\author{Robert G. Edwards (editor)}
\affiliation{Jefferson Lab}
%
\author{Jozef J. Dudek}
\affiliation{Jefferson Lab}
\affiliation{College of William and Mary}
%
\author{Michael Engelhardt}
\affiliation{New Mexico State University}
%
\author{Huey-Wen Lin}
\affiliation{Michigan State University}
%
\author{Stefan Meinel}
\affiliation{University of Arizona}
\affiliation{RIKEN BNL Research Center}
%
\author{Kostas Orginos}
\affiliation{Jefferson Lab}
\affiliation{College of William and Mary}
%
\author{Phiala Shanahan}
\affiliation{Massachusetts Institute of Technology}

\begin{abstract}

In 2018, the USQCD collaboration's Executive Committee organized several subcommittees to recognize future opportunities and formulate possible goals for lattice field theory calculations in several physics areas. The conclusion of these studies, along with community input, are presented in seven whitepapers \cite{Bazavov:2018qcd,Brower:2018qcd,Davoudi:2018qcd,Detmold:2018qcd,Joo:2018qcd,Kronfeld:2018qcd,Lehner:2018qcd}. Here, we discuss opportunities for lattice QCD calculations related to the structure and spectroscopy of hadrons and nuclei. An overview of recent lattice calculations of the structure of the proton and other hadrons is presented along with prospects for future extensions. Progress and prospects of hadronic spectroscopy and the study of resonances in the light, strange and heavy quark sectors is summarized. Finally, recent advances in the study of light nuclei from lattice QCD are addressed and the scope of future investigations that are currently envisioned is outlined.

\end{abstract}

\collaboration{USQCD Collaboration}

\maketitle

\tableofcontents
\newpage 

\section*{Executive summary}
\input executive

\section{Introduction}
\label{sec:intro}
\input intro

\section{Hadron Structure}
\label{sec:hadronstructure}
\input hadstruct

\section{Hadron Spectroscopy}
\label{sec:hadronspectroscopy}
\input hadspec

\section{Nuclear Spectroscopy, Interactions and Structure}
\label{sec:nuclear}
\input nuclear

\section{Future Opportunities}
\label{sec:future}
\input future

\begin{figure}
%    \includegraphics[width=\textwidth]{figs/chart}
    \vspace*{3cm}
\end{figure}

\bibliographystyle{apsrev4-1}
\bibliography{paper}

\end{document}
