% RGE is working on this...

%{\color{red} MOVE THIS BACK INTO THE RELEVANT SECTIONS I THINK - THERE IS STILL STUFF THERE IN PARTS AND IT FLOWS BETTER TO HAVE FUTURE INTERMINGLED}


\subsection{Hadron Structure}

%...Precision nucleon form factors and charge radii...
Understanding the structure of the proton and other hadrons is an important facet of  nuclear science and  has led to revolutionary discoveries over the last 70 years, including that of QCD itself.  
The discrepancy in the experimental measurements of the charge radius of the proton~\cite{Antognini:1900ns} has spurred a flurry of new measurements and phenomenological estimates. Current lattice QCD determinations of the proton radius are challenging since the derivative which defines the radius is extracted from modeling the $Q^2$ dependence of form factors from calculations at the discrete values of momentum accessible in a finite lattice volume.  Indeed, the problems encountered mirror those in electron-scattering experiments, where the form factor is computed for a discrete, albeit closely spaced, set of finite $Q^2$, and the need to include dispersive methods in the analysis of the form factors over the values of $Q^2$ probed in experiment has been emphasised~\cite{Alarcon:2018irp}.

A method to avoid these uncertainties involves the computation of coordinate-weighted moments of currents, without the need to model the $Q^2$ dependence~\cite{Bouchard:2016gmc}. Another recent approach~\cite{Detmold:2018ptb} introduces a mass splitting between the up and down quarks, allowing to access time-like as well as space-like four-momentum transfers close to $Q^2=0$. These calculations with statistical precision on the order of a few percent, and at the physical quark mass and several lattice spacings, are conceptually straightforward and achievable in the near term. The inclusion of disconnected diagrams would allow the access to the proton and neutron charge radii directly.

%...Nucleon electromagnetic form-factors and large momentum transfers...
The nucleon electric and magnetic form factors $G_{E,M}(Q^2)$ describe the distribution of charge and
magnetization inside the
nucleon~\cite{Burkardt:2000za,Burkardt:2002hr,Miller:2007uy,Carlson:2007xd} as a function of the momentum, $Q$, carried by the photon probe.
They have been extensively studied since the dawn of
accelerator technology, offering the first experimental evidence for composite structure 
of nucleons~\cite{Hofstadter:1955ae}, as well as the first
determination of the proton radius~\cite{Chambers:1956zz},
and remain an area of active experimental and theoretical research. The high-momentum limit yields a ``high-resolution'' picture of the nucleon, and is a subject of experiments at JLab.

Calculations involving hadrons with large momentum introduce unique challenges in LQCD: in the Breit frame one 
has to study nucleon states with high momentum 
resulting in statistical noise as well as excited state contributions due to the
shrinking energy gap between the nucleon states.
New techniques, such as momentum smearing~\cite{Bali:2016lva,Syritsyn:2017jrc}, have been shown to improve
the signal for boosted nucleon correlators by a factor of at least 10. The efficacy of the approach suggests precision calculations of nucleon form-factors are achievable, up to a few GeV$^2$ in a few years.


%...flavor dependence of PDF-s...
The recent development of techniques allowing for the extraction of the Bjorken-$x$ dependence of quark and gluon distribution functions directly from Euclidean space calculations~\cite{Ji:2001wha} has opened the door to a new age of hadronic and nuclear physics calculations. In particular, these methods allow, in principle,  the extraction of the large-$x$ dependence of quark and glue distributions that are a subject of the 12 GeV upgrade at JLab, and the small-$x$ dependence under study at RHIC experiment at BNL and the proposed EIC facility. Near-term  lattice calculations will establish the techniques and quantify systematic uncertainties for LQCD studies of the $x$-dependence of PDFs within systems like the pion, kaon and the nucleon. Establishing the flavor dependence of such parton distribution functions is also a near term goal. However, accessing the small-$x$ dependence, where the glue is expected to dominate, is a challenging goal, as naively one expects that very large lattice sizes with small lattice spacings are required. An intermediate approach could use anisotropic lattices to decrease the lattice spacing in a spatial direction.%, thus allow for a fine resolution in space and approach to a pertubative scaling regime.



\subsection{Hadron Spectroscopy}

% ...Resonance determination in the low energy sectors of QCD...
One goal of the spectroscopy program is establishing the branching fractions for decays of hadrons, including putative exotic mesons. It is these decay couplings that can inform and confront experimental analyses, such as those within GlueX and CLAS12. A target within the next few years is establishing the spectrum of the low-lying scalar, vector and tensor resonances in the physical limit of QCD. While these calculations must be mindful of potential three body decays, they are achievable within the next few years using resources available on leadership computing resources as well as USQCD resources.

Targeting exotic meson decays is more challenging, particularly because three-body decays might well be important. First calculations will necessarily need to use unphysically large quark masses where three-body thresholds are pushed higher in the spectrum and away from the resonance region of interest.
These initial calculations are tractable in the near term. However, the inclusion of three-body decays within searches for exotics is more challenging. The computational cost can be addressed with anticipated improvements of algorithms, but the understanding of how three- and higher-body decay channels can be included is more a conceptual question that needs to be addressed.


%...Radiative transition rates and form-factors...
Electromagnetic radiative transitions provide a probe into the structure of resonant states, and while challenging, are experimentally accessible. 
One notable target is the photo excitation of exotic mesons from pion exchange off the proton, the experimental production mechanism for the GlueX experiment. Thus, the extraction of the photo-production rate of exotic mesons is an important target for lattice calculations as they can inform the analysis of on-going experiments. The analytical formalism for the study of composite states exists~\cite{Briceno:2015tza}, and first studies have been carried out; however, the analytic formalism is not in place for systems with three body decays. To avoid such complications, first calculations will proceed at unphysical pion masses;  these studies are achievable in the near term. A successful extraction of a photo-coupling will be an important step for phenomenology.

Beyond reproducing experimentally accessible reactions, lattice calculations can also investigate physically relevant quantities that cannot be determined experimentally. First such calculations will include the elastic form factors of hadronic resonances. Encoded in these is structural information, which will give further insight into the true nature of these states, e.g., their size and shape. The computational aspects are manageable, while the analytic formalism for such systems is maturing~\cite{Briceno:2015tza,Baroni:2018unx}.


%...Nature of near-threshold charmonium resonances...
A compelling question that remains unanswered in the charmonium sector is the nature of the `XYZ' resonances. They often appear in close proximity to thresholds leading to wide-ranging speculations that some of these might be `molecular' in origin. These questions can be tested by studying the response of the state to variations of the position of the threshold induced by changing the light and charm quark masses. In addition, the behavior of radiative transition decays and form-factors, including the calculation of charge radii, will provide valuable insight into their nature. Necessarily, the calculations will involve a range of light and charm quark masses, not necessarily at their physical values.  These computations are relatively straightforward, and achievable within the next few years. However, some of the systems, such as the $X(3872)$, are very close to threshold thus potentially requiring the inclusion of isospin breaking effects as well as high statistical precision.


\subsection{Nuclear Interactions}

Since nuclei make up the majority of the visible matter in the Universe, understanding their emergence from the underlying theory of the strong interaction is a fundamental challenge bridging nuclear and particle physics. Large-scale numerical calculations will allow us to address this challenge and achieve a quantitative connection between the Standard Model and nuclear phenomenology, opening new directions in our quest to interpret the complexities of nuclear physics and supporting experimental efforts to use nuclei to reveal fundamental aspects of nature.

%...Light nuclei at physical quark masses...
Determinations of the finite-volume energy levels of few nucleon systems constrain the two- and three- nucleon forces in EFT methods~\cite{Barnea:2013uqa}, thereby enabling predictions of the properties and interactions of larger nuclei. Calculations of the spectrum of light nuclei with atomic number $A\le 4$ with quark masses close to the physical limit are achievable in the near term, given ${\cal O}(10^5)$ quark propagator sources, coupled with techniques such as the variational method~\cite{Michael:1985ne}, improved estimators \cite{Beane:2014oea}, along with signal-to-noise optimization methods~\cite{Detmold:2014hla}, all to enhance the statistical signal. While not fully resolving all systematic uncertainties, however, these calculations are expected to represent a significant step forward in showing how nuclei emerge from the intricacies of Standard Model dynamics.

%...p-shell nuclei..
The extension to the larger $p$-shell nuclei will be important in the future as they are more sensitive to three body nuclear forces than lighter nuclei. They also present a new level of challenge for calculations as their structure is more complicated. These systems will require significantly more sophisticated constructions of interpolating operators such as determinant contraction methods~\cite{Detmold:2012eu,Vachaspati:2014bda}. A large number of quark propagators will be required and the resulting cost of the contractions of these propagators will be large as well. Current development efforts are underway that will allow  first tests of the efficacy of the approach, initially at unphysical pion masses, in the near term.


%...Matrix elements of nuclei...
As outlined in this and accompanying white-papers, there are strong phenomenological motivations for studies of the scalar, axial and tensor current matrix elements of light nuclei. Calculations of the matrix elements in light nuclei up to $A=4$ \cite{Winter:2017bfs} at close-to-physical values of the quark masses, will constrain necessary inputs for current and future experiments using nuclear targets for searches for physics beyond the Standard Model. The statistical requirements for each calculation will depend on the magnitude of the signal in each channel, which is currently unknown. First studies are expected to proceed with the scalar current as it appears to have the strongest nuclear effects. Connected contributions can use external field techniques~\cite{Savage:2016kon,Shanahan:2017bgi,Tiburzi:2017iux} while disconnected contributions are expected to be important. These calculations will establish the baseline for the statistics required, and while computationally challenging, are achievable in a few years time.



