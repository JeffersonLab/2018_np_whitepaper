
 
\hide{Beyond the physics of single hadrons described in the previous sections, the wealth of complexity embodied in the nuclear landscape emerges from QCD and the other forces of the SM. The first steps in addressing this complexity from LQCD have been made over the last decade, and we anticipate that LQCD will be an increasingly important part of nuclear theory in the coming years. Since the SM forms the foundation of nuclear physics, LQCD can be used to study the forces that bind nucleons into nuclei and govern their interaction, as well as to investigate how nuclear systems interact with external electroweak and possible physics beyond the Standard Model.}

\subsection{Nuclear spectroscopy}

Determining the ground state energies and binding energies of light nuclei is a central challenge. The very first QCD calculations of nuclei less than a decade old and significant advances in the study of nuclear systems have occurred over the last five years.
The first calculations of bound systems with baryon number $A\ge2$ were of the doubly-strange $H$-dibaryon system at unphysical quark masses
\cite{Beane:2010hg,Inoue:2010es,Beane:2011xf}. Additional calculations have ensued \cite{Beane:2012vq,Yamazaki:2012hi,Yamazaki:2015asa}, with almost physical quark mass calculations being currently performed by the PACS-CS collaboration \cite{}. Of particular note

Nuclear systems are particularly challenging for LQCD for multiple reasons. As emphasised by Parisi and Lepage \cite{Lepage:1989hd,Parisi:1983ae,Hamber:1983vu}, single baryon correlation functions exhibit a signal-to-noise ratio that degrades exponentially with the temporal separation. FOr nuclear systems, the problem only becomes more challenging \cite{Beane:2009kya,Beane:2009gs}. Extraction of the eigen-energies is consequently challenging. A number of methods have been developed that aim to ameliorate this issue, either defining better-behaved estimators \cite{Beane:2014oea,Wagman:2017gqi,Wagman:2017xfh,Wagman:2016bam} or new analysis strategies that optimize the ratio of signal to noise \cite{Detmold:2014hla}. None of these methods has completely solved the noise problem, but have proved sufficient for studies of the lightest few nuclei. 

The first calculations of bound nuclear states in QCD have been performed by the NPLQCD collaboration \cite{} and the PACS-CS collaboration (bound states have also been found using HAL potential method \cite{} based on Refs.~\cite{}, although it is only recently that systematics have begun to be addressed in this method \cite{}). Bound states have been studied over a range of quark masses and atomic numbers with a recent review of these results is presented in Ref.~\cite{}. These calculations have been used as input in effective field theory analyses that extend the reach of LQCD calculations to larger nuclei \cite{Barnea:2013uqa,MORE}.


There are many opportunities for increased effort in this area as well as many technical challenges that exist in going to larger systems and performing calculations at the physical quark masses. 
Extending existing calculations to even moderately larger $A$ will have significant impact as nuclei that require $p$-shell configurations become accessible. These systems depend on more complicated aspects of hte nuclear forces than the $A\le4$ nuclei that have been studied and new lattice calculations will be useful in constraining different spin and isospin components of these forces. Larger nuclei also exhibit interesting collective effects such as halo structures (eg,  ${}^{6,8}$He), cluster structures (eg ${}^{8}$Be, ${}^{12}$C) and deformations that would be very instructive to see emerge from LQCD calculations. Additionally, LQCD offers the possibility of investigating nuclei away from the physical quark masses, or for different gauge and fermion content of the theory, as an intellectual pursuit of its own in which questions related to the naturalness of nuclear physics \cite{Orginos:2015aya}. Calculations in this direction using $N_f=N_c=2$ appear in Refs.~\cite{Detmold:2014qqa,Detmold:2014kba}.

For bound nuclear systems there are two exponentially difficult algorithmic challenges that must be addressed. The complexity of contractions grows 
facorially with the system size, at least naively; calculations for $^4$He are naively $6!6!/2\sim 260,000$ more difficult than for a proton. Efforts to reduce these costs have enabled the progress described above and have proceeded via construction of enumerative \cite{Doi:2012xd,Gunther:2013xj} and recursive \cite{Detmold:2012eu} algorithms. 


\subsection{Nuclear Structure}
\label{sec:nuclearstructure}

Nuclei are complex objects and exploration of their structure from the underlying quark and gluon degrees of freedom offers challenges and opportunities for new calculations. The structure of 


Interactions of nuclei with electroweak probes provide much of the information we have about hadron structure phenomenologucally. 
The magnetic moments,  higher multiple moments and polarizabilities  enable a static picture of nuclei to be determined. The corresponding form factors 

Nuclear parton distributions provide further information on the substructure of nuclei and have historically been one of the 
most striking examples of an arena in which non-nucleonic degrees of freedom are important inside the nucleus. The EMC effect \cite{Aubert:1983XX}  

\begin{itemize}
	\item Explore nuclear modification of hadron structure. Modifications of charges, form factors, moments of PDFs  (derive the EMC effect from QCD). 
	Do this with complete flavour breakdown
	\item Determine spatial pictures of nuclei from ``charge distributions'' as Fourier transforms of various form factors. Radii
	\item Study $x$-dependent PDFs, gluonic aspects of hnuclear structure
	\item Exotic glue
\end{itemize}





\subsection{Nuclear interactions}

\begin{itemize}
	\item Scattering  phase shifts of $NN$ systems. validation of LQCD calculations at physical point. Understanding role of Coulomb, eg $a_{pp}$ vs $a_{np}$ vs $a_{nn}$ in the spin singlet channel.
	\item Scattering phase shifts for  hyperon-nucleon and hyperon-hyperon systems. Input to nuclear equation of state (EoS) for potential hyperonic matter in neutron stars; connection to NS-NS mergers and NICER.
	\item Three and four body forces. Experimental constrain in particular on $nnn$ interactions is poor but increasingly relevant in larger nuclei. Most direct constrain is from EFT matching to finite volume energy levels computed in LQCD. Multiple EFT groups using unphysical mass lattice calculations to do this right now:LANL, Hagen/ORNL (pion-full), Lovato/Pederiva, Van Kolck using different EFT formulations
	\item Interaction of light nuclei with electroweak probes: $np\to d\gamma$, $pp\to de^+\nu$, ... 
	Important to extract two, three body current contributions as one-body pieces presumably better known from single nucleon calculations/experiment.
	\item Double beta decay
\end{itemize}

For some specific nuclei such as germanium, single $\beta$ decay is energetically forbidden, but double $\beta$ decay is allowed. In the Standard Model, this decay occurs with the release of two electrons and two anti-neutrinos, conserving lepton number  ($2\nu\beta\beta$-decay). In many beyond the standard model scenarios, either with light Majorana neutrinos (particles that are their own antiparticles) or with other forms of lepton number non-conservation at high scales,  a second form of  double $\beta$ decay that involves no neutrinos in the final state ($0\nu\beta\beta$-decay) can occur. Observation of this latter process would be an unambiguous signal for new physics. Understanding of the implications of such an observation, as well as design of future experiments seeking this decay mode, requires understanding the relevant $\Delta I=2$ nuclear transition matrix elements. This is a challenging task and state-of-the-art nuclear theory calculations differ by an order of magnitude. LQCD offers the prospect of providing useful input in this area through calculations of the relevant matrix elements in light nuclei that can be used to control uncertainties in nuclear models. In the last two years, the $2\nu\beta\beta$ process has been studied in the $pp\to nn$ transition \cite{Tiburzi:2017iux,Shanahan:2017bgi},  the pionic matrix elements, $\langle \pi^+ | {\cal O} | \pi^- \rangle$, of $\Delta I =2$ short distance operators \cite{Nicholson:2018mwc}, and the $\pi^-\to \pi^+ e^- e^-$ and $\pi^-\pi^-\to e^-e^-$ transitions induced  by a light Majorane neutrino \cite{Feng:2018pdq,Murphy2018ICHEP}, have all been investigated for the first time. Future refinements of these calcualtions have the potential to significan




\subsection{Nuclear input for neutrino physics and fundamental symmetries}


Nuclei are used as targets in intensity frontier experiments that are trying to probe the neutrino sector and to look for physics beyond the SM. In particular, argon ($Z=18$) is the target material for a number of current neutrino experiments and will be the target for the upcoming Deep Underground Neutrino Experiment (DUNE), while a range of nuclei such as sodium ($Z=11$), xenon ($Z=54$) are used in dark matter direct detection experiments \cite{Undagoitia:2015gya}. Charge lepton flavor violation searches look for $\mu\to e$ conversion in the field of aluminium ($Z=13$) \cite{Albrecht:2013wet}, and precision isotope-shift spectroscopy experiments consider a wide range of nuclei ranging from hydrogen ($Z=1$) to ytterbium ($Z=70$) in  order to constrain new physics \cite{Delaunay:2016brc,Delaunay:2017dku}, both requiring knowledge of various nuclear matrix elements \cite{Chang:2017eiq}. Finally, double-$\beta$ decay (DBD) searches utilize heavy isotopes to search for lepton number violation through neutrinoless DBD \cite{DellOro:2016tmg,Engel:2016xgb,Nicholson:2016byl,Shanahan:2017bgi,Tiburzi:2017iux}.

All of the techniques discussed above in the study of nuclear spectroscopy, structure and interactions are applicable in these areas, and calculations are being actively pursued. We leave a full discussion of these topics to the two companion USQCD whitepapers on Neutrino-Nucleus Interactions and on Fundamental Symmetries.



