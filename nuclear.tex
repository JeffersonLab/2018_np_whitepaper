
 
\hide{Beyond the physics of single hadrons described in the previous sections, the wealth of complexity embodied in the nuclear landscape emerges from QCD and the other forces of the SM. The first steps in addressing this complexity from LQCD have been made over the last decade, and we anticipate that LQCD will be an increasingly important part of nuclear theory in the coming years. Since the SM forms the foundation of nuclear physics, LQCD can be used to study the forces that bind nucleons into nuclei and govern their interaction, as well as to investigate how nuclear systems interact with external electroweak and possible physics beyond the Standard Model.}

\subsection{Nuclear spectroscopy}

\hide{Determining the ground state energies and binding energies of light nuclei is a central challenge. The very first QCD calculations of nuclei less than a decade old and significant advances in the study of nuclear systems have occurred over the last five years.
The first calculations of bound systems with baryon number $A\ge2$ were of the doubly-strange $H$-dibaryon system at unphysical quark masses
\cite{Beane:2010hg,Inoue:2010es,Beane:2011xf}. Additional calculations have ensued \cite{Beane:2012vq,Yamazaki:2012hi,Yamazaki:2015asa}, with almost physical quark mass calculations being currently performed by the PACS-CS collaboration \cite{}

Nuclear systems are particularly challenging for LQCD for multiple reasons. As emphasised by Parisi and Lepage \cite{Lepage:1989hd,Parisi}, single baryon correlation functions exhibit a signal-to-noise ratio that degrades exponentially with the temporal separation. Extraction of the eigen-energies is consequently challenging. A number of methods have been developed that aim to ameliorate this issue, either defining better-behaved estimators \cite{Beane:2014oea,Wagman:2017gqi,Wagman:2017xfh,Wagman:2016bam} or new analysis strategies that optimize the ratio of signal to noise \cite{Detmold:2014hla}. None of these methods has completely solved the noise problem, but have proved sufficient for studies of light nuclei.

}



\begin{itemize}
	\item Spectroscopy of light nuclei: physical mass calculations of $A<5$ nuclei
	
	\item Exploratory calculations of larger nuclei with strong motivation for $^6$Li as it is spin-1, also for other $p$-shell nuclei as they probe aspects of nuclear forces 
	
	\item Try to look at states that have complex structures at the physical masses (eg halo for ${}^6$He, deformations) though presumably many small effects important 
	
	\item Understand quark mass dependence of nuclear physics
	
	\item FV energy levels matching to effective field theories in FV. 
	
\end{itemize}






\subsection{Nuclear interactions}

\begin{itemize}
	\item Scattering  phase shifts of $NN$ systems. validation of LQCD calculations at physical point. Understanding role of Coulomb, eg $a_{pp}$ vs $a_{np}$ vs $a_{nn}$ in the spin singlet channel.
	\item Scattering phase shifts for  hyperon-nucleon and hyperon-hyperon systems. Input to nuclear equation of state (EoS) for potential hyperonic matter in neutron stars; connection to NS-NS mergers and NICER.
	\item Three and four body forces. Experimental constrain in particular on $nnn$ interactions is poor but increasingly relevant in larger nuclei. Most direct constrain is from EFT matching to finite volume energy levels computed in LQCD. Multiple EFT groups using unphysical mass lattice calculations to do this right now:LANL, Hagen/ORNL (pion-full), Lovato/Pederiva, Van Kolck using different EFT formulations
	\item Interaction of light nuclei with electroweak probes: $np\to d\gamma$, $pp\to de^+\nu$, ... 
	Important to extract two, three body current contributions as one-body pieces presumably better known from single nucleon calculations/experiment.
\end{itemize}


\subsection{Nuclear Structure}
\label{sec:nuclearstructure}
\begin{itemize}
	\item Explore nuclear modification of hadron structure. Modifications of charges, form factors, moments of PDFs  (derive the EMC effect from QCD). 
	Do this with complete flavour breakdown
	\item Determine spatial pictures of nuclei from ``charge distributions'' as Fourier transforms of various form factors. Radii
	\item Study $x$-dependent PDFs, gluonic aspects of hnuclear structure
	\item Exotic glue
\end{itemize}



\subsection{Nuclear input for neutrino physics and fundamental symmetries}

\begin{itemize}
	\item nuclear matrix elements for experiment interpretation (DM searches, nu-nucleus scattering,…)
	\item dark matter matrix elements [passing connection to other doc on Fundamental Symmetries], $\mu 2e$ matrix elements
	\item Precision isotope shift spectroscopy 
	\item Primarily link to other white papers
\end{itemize}

\hide{
Nuclei are used as targets in intensity frontier experiments that are trying to probe the neutrino sector and to look for physics beyond the SM. In particular, argon ($Z=18$) is the target material for a number of current neutrino experiments and will be the target for the upcoming Deep Underground Neutrino Experiment (DUNE), while a range of nuclei such as sodium ($Z=11$), xenon ($Z=54$) are used in dark matter direct detection experiments \cite{Undagoitia:2015gya}. Charge lepton flavor violation searches look for $\mu\to e$ conversion in the field of aluminium ($Z=13$) \cite{Albrecht:2013wet}, and precision isotope-shift spectroscopy experiments consider a wide range of nuclei ranging from hydrogen ($Z=1$) to ytterbium ($Z=70$) in  order to constrain new physics \cite{Delaunay:2016brc,Delaunay:2017dku}, both requiring knowledge of various nuclear matrix elements \cite{Chang:2017eiq}. Finally, double-$\beta$ decay (DBD) searches utilize heavy isotopes to search for lepton number violation through neutrinoless DBD \cite{DellOro:2016tmg,Engel:2016xgb,Nicholson:2016byl,Shanahan:2017bgi,Tiburzi:2017iux}.

All of the techniques discussed above in the study of nuclear spectroscopy, structure and interactions are applicable in these areas, and calculations are being actively pursued. We leave a full discussion of these topics to the two companion USQCD whitepapers on Neutrino-Nucleus Interactions and on Fundamental Symmetries.

}

